\section{Завдання}
Для наведених прикладів ланцюжків побудувати правила граматики.
Перевірити правильність складання правил за допомогою виведення.
Перевірити наявність непродуктивних та недосяжних символів:


\section{Побудова правил граматики}
\begin{enumerate}
    \item \verb|var i:array[1..n] of real; j:array[1..n] of i:integer|
    \item \verb|var k:array[1..n] of i:integer|
\end{enumerate}

\subsection{Опис граматики}

% \begin{enumerate}
%     \item  \verb|I| $\to$ \verb|var SR|
%     \item  \verb|R| $\to$ \verb|;SR or $|
%     \item  \verb|S| $\to$ \verb|N:array[1..n] of A|
%     \item  \verb|N| $\to$ \verb|i or j or k|
%     \item  \verb|A| $\to$ \verb|N:integer or real|
% \end{enumerate}

\begin{enumerate}
    \item  \verb|I| $\to$ \verb|var SR|
    \item  \verb|R| $\to$ \verb|;SR or $|
    \item  \verb|S| $\to$ \verb|V of A|
    \item  \verb|V| $\to$ \verb|N:T|
    \item  \verb|N| $\to$ \verb|i or j or k|
    \item  \verb|T| $\to$ \verb|array[1..n] or integer|
    \item  \verb|A| $\to$ \verb|V or real|
\end{enumerate}

\newpage

% \subsection{Перевірка граматики}
% Ланцюжок: \verb|var i:array[1..n] of real; j:array[1..n] of i:integer|
% \begin{itemize}
%     \item[]  I $\xrightarrow{1}$ \verb|var SR|
%     \item[]  $\xrightarrow{3}$ \verb|var N:array[1..n] of AR|
%     \item[]  $\xrightarrow{4.1}$ \verb|var i:array[1..n] of AR|
%     \item[]  $\xrightarrow{5.2}$ \verb|var i:array[1..n] of realR|
%     \item[]  $\xrightarrow{2.1}$ \verb|var i:array[1..n] of real; SR|
%     \item[]  $\xrightarrow{3}$ \verb|var i:array[1..n] of real; N:array[1..n] of AR|
%     \item[]  $\xrightarrow{4.2}$ \verb|var i:array[1..n] of real; j:array[1..n] of AR|
%     \item[]  $\xrightarrow{5.1}$ \verb|var i:array[1..n] of real; j:array[1..n] of N:integerR|
%     \item[]  $\xrightarrow{4.1}$ \verb|var i:array[1..n] of real; j:array[1..n] of i:integerR|
%     \item[]  $\xrightarrow{2.2}$ \verb|var i:array[1..n] of real; j:array[1..n] of i:integer|
% \end{itemize}

\subsection{Перевірка граматики}
Ланцюжок: \verb|var i:array[1..n] of real; j:array[1..n] of i:integer|
\begin{itemize}
    \item[]  I $\xrightarrow{1}$ \verb|var SR|
    \item[]  I $\xrightarrow{3}$ \verb|var V of AR|
    \item[]  I $\xrightarrow{4}$ \verb|var N:T of AR|
    \item[]  I $\xrightarrow{5.1}$ \verb|var i:T of AR|
    \item[]  I $\xrightarrow{6.1}$ \verb|var i:array[1..n] of AR|
    \item[]  I $\xrightarrow{7.2}$ \verb|var i:array[1..n] of realR|
    \item[]  I $\xrightarrow{2.1}$ \verb|var i:array[1..n] of real;SR|
    \item[]  I $\xrightarrow{3}$ \verb|var i:array[1..n] of real; V of AR|
    \item[]  I $\xrightarrow{4}$ \verb|var i:array[1..n] of real; N:T of AR|
    \item[]  I $\xrightarrow{5.2}$ \verb|var i:array[1..n] of real; j:T of AR|
    \item[]  I $\xrightarrow{6.1}$ \verb|var i:array[1..n] of real; j:array[1..n] of AR|
    \item[]  I $\xrightarrow{7.1}$ \verb|var i:array[1..n] of real; j:array[1..n] of VR|
    \item[]  I $\xrightarrow{4}$ \verb|var i:array[1..n] of real; j:array[1..n] of N:TR|
    \item[]  I $\xrightarrow{5.1}$ \verb|var i:array[1..n] of real; j:array[1..n] of i:TR|
    \item[]  I $\xrightarrow{6.2}$ \verb|var i:array[1..n] of real; j:array[1..n] of i:integerR|
    \item[]  I $\xrightarrow{2.2}$ \verb|var i:array[1..n] of real; j:array[1..n] of i:integer|
\end{itemize}

\newpage
\subsection{Перевірка на непродуктивність}
% \begin{enumerate}
%     \item  R N
%     \item  R N I
%     \item  R N I A
%     \item  R N I A S
%     \item  нема непродуктивних символів
% \end{enumerate}
\begin{enumerate}
    \item  R N T
    \item  R N T V
    \item  R N T V A
    \item  R N T V A S
    \item  R N T V A S I
\end{enumerate}
нема непродуктивних символів

\subsection{Перевірка на недосяжність}
% \begin{enumerate}
%     \item  I
%     \item  I S R
%     \item  I S R N A
%     \item  нема недосяжних символів
% \end{enumerate}
\begin{enumerate}
    \item  I
    \item  I S R
    \item  I S R V A
    \item  I S R V A N T
\end{enumerate}
нема недосяжних символів

\subsection{Тип граматики}
Граматика не є простою. Граматика є LL(1) граматикою.

\newpage
\section{Побудова магазинного автомату}
\subsection{Побудова функції ПЕРШ}
% \begin{itemize}
%     \item  ПЕРШ(\verb|I| $\to$ \verb|var SR|) = \{\verb|var|\}
%     \item  ПЕРШ(\verb|R| $\to$ \verb|;SR|) = \{\verb|;|\}
%     \item  ПЕРШ(\verb|R| $\to$ \verb|$|) = \{\verb|$|\}
%     \item  ПЕРШ(\verb|S| $\to$ \verb|N:array[1..n] of A|) = ПЕРШ(\verb|N|) = \{\verb|i, j, k|\}
%     \item  ПЕРШ(\verb|N| $\to$ \verb|i|) = \{\verb|i|\}
%     \item  ПЕРШ(\verb|N| $\to$ \verb|j|) = \{\verb|j|\}
%     \item  ПЕРШ(\verb|N| $\to$ \verb|k|) = \{\verb|k|\}
%     \item  ПЕРШ(\verb|A| $\to$ \verb|N:integer|) = \{\verb|i, j, k|\}
%     \item  ПЕРШ(\verb|A| $\to$ \verb|real|) = \{\verb|real|\}
% \end{itemize}

\begin{itemize}
    \item  ПЕРШ(\verb|I| $\to$ \verb|var SR|) = \{\verb|var|\}
    \item  ПЕРШ(\verb|R| $\to$ \verb|;SR|) = \{\verb|;|\}
    \item  ПЕРШ(\verb|R| $\to$ \verb|$|) = \{\verb|$|\}
    \item  ПЕРШ(\verb|S| $\to$ \verb|V of A|) = ПЕРШ(\verb|V|) = ПЕРШ(\verb|N|) = \{\verb|i, j, k|\}
    \item  ПЕРШ(\verb|V| $\to$ \verb|N:T|) = ПЕРШ(\verb|N|) = \{\verb|i, j, k|\}
    \item  ПЕРШ(\verb|N| $\to$ \verb|i|) = \{\verb|i|\}
    \item  ПЕРШ(\verb|N| $\to$ \verb|j|) = \{\verb|j|\}
    \item  ПЕРШ(\verb|N| $\to$ \verb|k|) = \{\verb|k|\}
    \item  ПЕРШ(\verb|T| $\to$ \verb|array[1..n]|) = \{\verb|array|\}
    \item  ПЕРШ(\verb|T| $\to$ \verb|integer|) = \{\verb|integer|\}
    \item  ПЕРШ(\verb|A| $\to$ \verb|V|) = ПЕРШ(\verb|V|) = ПЕРШ(\verb|N|) = \{\verb|i, j, k|\}
    \item  ПЕРШ(\verb|A| $\to$ \verb|real|) = \{\verb|real|\}
\end{itemize}

\subsection{Побудова функції СЛІД}
\begin{itemize}
    \item  СЛІД(\verb|I|) = \{\verb|$|\}
    \item  СЛІД(\verb|R|) = \{\verb|$|\}
    \item  СЛІД(\verb|S|) = \{\verb|;|\}
    \item  СЛІД(\verb|V|) = \{\verb|of|\}
    \item  СЛІД(\verb|N|) = \{\verb|:|\}
    \item  СЛІД(\verb|T|) = \{\verb|$|\}
    \item  СЛІД(\verb|A|) = \{\verb|$|\}
\end{itemize}

% \subsection{Побудова функції ВИБІР}
% \begin{itemize}
%     \item  ВИБІР(\verb|I| $\to$ \verb|var SR|) = ПЕРШ(1) = \{\verb|var|\}
%     \item  ВИБІР(\verb|R| $\to$ \verb|;SR|) = ПЕРШ(2.1) = \{\verb|;|\}
%     \item  ВИБІР(\verb|R| $\to$ \verb|$|) = СЛІД(R) = \{\verb|$|\}
%     \item  ВИБІР(\verb|S| $\to$ \verb|N:array[1..n] of A|) = ПЕРШ(3) = \{\verb|i, j, k|\}
%     \item  ВИБІР(\verb|N| $\to$ \verb|i|) = ПЕРШ(4.1) = \{\verb|i|\}
%     \item  ВИБІР(\verb|N| $\to$ \verb|j|) = ПЕРШ(4.2) = \{\verb|j|\}
%     \item  ВИБІР(\verb|N| $\to$ \verb|k|) = ПЕРШ(4.3) = \{\verb|k|\}
%     \item  ВИБІР(\verb|A| $\to$ \verb|N:integer|) = ПЕРШ(1) = \{\verb|i, j, k|\}
%     \item  ВИБІР(\verb|A| $\to$ \verb|real|) = СЛІД(R) = \{\verb|real|\}
% \end{itemize}


% \subsection{Побудова команд розпізнавача}
% \begin{enumerate}
%     \item  f (\verb|s, var, I |) = \verb|(s, RS)|\
%     \item  f (\verb|s, ;, R |) = \verb|(s, RS)|\
%     \item  f* (\verb|s, $, R |) = \verb|(s, $)|\



%     \item  f* (\verb|s, S, A |) = \verb|(s, )X(F )|\
%     \item  f* (\verb|s, C, A |) = \verb|(s, )X(F )|\
%     \item  f (\verb|s, (, A |) = \verb|(s, )X )|\
%     \item  f (\verb|s, *, R |) = \verb|(s, RA)|\
%     \item  f* (\verb|s, ), R |) = \verb|(s, $)|\
%     \item  f (\verb|s, (, ( |) = \verb|(s, $)|\
%     \item  f (\verb|s, ), ) |) = \verb|(s, $)|\
%     \item  f (\verb|s, X, X |) = \verb|(s, $)|\
%     \item  f* (\verb|s, $, h0 |) = \verb|(s, $)|\
% \end{enumerate}

% \newpage
% \subsection{Перевірка команд розпізнавача}
% Ланцюжок: \verb|#define C(X) (S(X)*(X))|

% \begin{itemize}
%     \item[]  (s, \quad \verb|#define C(X) (S(X)*(X))$|,           \quad $h_{0}$\verb|I|)      $\vdash$ 1
%     \item[]  (s, \quad \verb|C(X) (S(X)*(X))$|,           \quad $h_{0}$\verb|)RA()X(F |)      $\vdash$ 3
%     \item[]  (s, \quad \verb|(X) (S(X)*(X))$|,           \quad $h_{0}$\verb|)RA()X( |)        $\vdash$ 9
%     \item[]  (s, \quad \verb|X) (S(X)*(X))$|,           \quad $h_{0}$\verb|)RA()X |)          $\vdash$ 11
%     \item[]  (s, \quad \verb|) (S(X)*(X))$|,           \quad $h_{0}$\verb|)RA() |)            $\vdash$ 10
%     \item[]  (s, \quad \verb|(S(X)*(X))$|,           \quad $h_{0}$\verb|)RA( |)               $\vdash$ 9
%     \item[]  (s, \quad \verb|S(X)*(X))$|,           \quad $h_{0}$\verb|)RA |)                 $\vdash$ 4
%     \item[]  (s, \quad \verb|S(X)*(X))$|,           \quad $h_{0}$\verb|)R)X(F |)              $\vdash$ 2
%     \item[]  (s, \quad \verb|(X)*(X))$|,           \quad $h_{0}$\verb|)R)X( |)                $\vdash$ 9
%     \item[]  (s, \quad \verb|X)*(X))$|,           \quad $h_{0}$\verb|)R)X |)                  $\vdash$ 11
%     \item[]  (s, \quad \verb|)*(X))$|,           \quad $h_{0}$\verb|)R) |)                    $\vdash$ 10
%     \item[]  (s, \quad \verb|*(X))$|,           \quad $h_{0}$\verb|)R |)                      $\vdash$ 7
%     \item[]  (s, \quad \verb|(X))$|,           \quad $h_{0}$\verb|)RA |)                      $\vdash$ 6
%     \item[]  (s, \quad \verb|X))$|,           \quad $h_{0}$\verb|)R)X |)                      $\vdash$ 11
%     \item[]  (s, \quad \verb|))$|,           \quad $h_{0}$\verb|)R) |)                        $\vdash$ 10
%     \item[]  (s, \quad \verb|)$|,           \quad $h_{0}$\verb|)R |)                          $\vdash$ 8
%     \item[]  (s, \quad \verb|)$|,           \quad $h_{0}$\verb|) |)                           $\vdash$ 10
%     \item[]  (s, \quad \verb|$|,           \quad $h_{0}$\verb||)                              $\vdash$ 12
%     \item[]  (s, \quad \verb|$|,           \quad \verb|$|) - успішне розпізнавання
% \end{itemize}

% Після перевірки ми дійшли до заключної конфігурації, отже рядок належить до граматики.


