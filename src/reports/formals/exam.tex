\section{Алгоритм зсув-згортка. LR(1) – граматики. Побудова  керуючої таблиці МП автомата для граматик, що містять анулюючі правила. }

\text{Алгоритм зсув-згортка} - цей алгоритм призначений для правильної роботи висхідних розпізнавачів, головною операцією є операція згортки. Для алгоритма потрібні дві таблиці (переходів та керуюча). \\

Таблиця переходів будується з граматичних входжень (рядки) та граматичних символів (стовпці). Кожному граматичному входженню відповідає один рядок таблиці, а кожному граматичному символу – один
стовпець.\\

Керуюча таблиця будується з граматичних входжень (рядки) та термінальних символів  (стовпці). У цій таблиці кожному пересіченню стовпця та рядка відповідає певна операція, є дві головні операції алгоритму: П - перенос, З(№) - згортка за певним правилом, та також другорядні операції для позначення дій, що здійснють передачу на вихід результатів роботи розпізнавача: В - відкинути, Д - допустити.\\

Алгоритм роботи:

\begin{enumerate}
  \item Прочитати черговий символ вхідного ланцюжка.
  \item Прочитати символ стану, що знаходиться у вершині магазина.
  \item Прочитати значення елемента керуючої таблиці, що знаходиться на пересічені певного рядку та стовпця.
  \item Якщо прочитане значення є В чи Д, то роботу варто закінчити, оскільки результат отриманий.
  \item Якщо прочитане значення визначає операцію П, то прочитати в таблиці переходів елемент що знаходится на пересічені відповідного рядка і стовпця. Записати отриманий символ у магазин.
  \item Якщо прочитане значення визначає операцію З(№) в лівий нетермінальний символ граматики за якою іде згортка (нехай цей символ буде K), то прочитати в таблиці переходів елемент Kij, що
  знаходиться в стовпці K і рядку, який відповідає верхньому символу магазина, що не приймає участі у згортці. Записати Kij у магазин і перейти до п.1.​
\end{enumerate}

\section{Побудова детермінованого спадного розпізнавача для простої граматики. }
Побудова розпізнавача передбачає протиставлення кожному правилу граматики команди розпізнавача. Відповідно до загального способу побудови розпізнавача кожному правилу розділеної граматики, що мають вигляд:
\[
A \to a \alpha,
\]
де $\alpha$ – ланцюжок символів повного словника, а $a$ належить термінальному словнику, потрібно поставити у відповідність команду:
\[
f(s_0, a, A) = (s_0, \alpha'),
\]
яка визначає такт роботи розпізнавача зі зсувом вхідної головки, і у якій $\alpha'$ являє собою дзеркальне відображення ланцюжка $\alpha$.\\

Крім того, варто врахувати, що термінальні символи можуть бути розташовані в правих частинах правил не тільки на самій лівій позиції. Для таких терміналів необхідно побудувати команди вигляду:
\[
f(s_0, b, b) = (s_0, \$).
\]

Для переходу в заключний стан додамо ще одне правило:
\[
f(s_0, \$, h_0) = (s_1, \$),
\]
а як початкову конфігурацію розпізнавача приймемо звичайний вираз:
\[
(s_0, \alpha, h_0I),
\]
де $I$ – початковий символ граматики, а $\alpha$ – заданий вхідний ланцюжок.\\

\textbf{Приклад побудови команд розпізнавача для розділеної граматики:}
\[
R = \{I \to abB, \; I \to bBbI, \; B \to a, \; B \to bB\}.
\]
\textbf{Команди:}
\begin{align*}
f(s_0, a, I) &= (s_0, Bb), \\
f(s_0, a, B) &= (s_0, \$), \\
f(s_0, b, I) &= (s_0, IbB), \\
f(s_0, b, B) &= (s_0, B), \\
f(s_0, b, b) &= (s_0, \$), \\
f(s_0, \$, h_0) &= (s_0, \$).
\end{align*}



