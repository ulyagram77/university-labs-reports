\section{Завдання}
Для наведених прикладів ланцюжків побудувати правила граматики.
Перевірити правильність складання правил за допомогою виведення.
Перевірити наявність непродуктивних та недосяжних символів:


\section{Побудова правил граматики}
\begin{enumerate}
    \item \verb|#define S(X) ((X) * (X))|
    \item \verb|#define C(X) (S(X)*(X))|
    \item \verb|#define C(X) ( )|
    \item \verb|#define S(X) ( )|
\end{enumerate}

\subsection{Опис граматики}
\begin{enumerate}
    \item  \verb|I| $\to$ \verb|#define F(X) (AR)|
    \item  \verb|F| $\to$ \verb|S or C|
    \item  \verb|A| $\to$ \verb|F(X) or (X)|
    \item  \verb|R| $\to$ \verb|*AR or $|
\end{enumerate}

\newpage
\subsection{Перевірка граматики}
Приклад для перевірки: \verb|#define C(X) (S(X)*(X))|
\begin{itemize}
    \item[]  I $\xrightarrow{1}$ \verb|#define F(X) (AR)|
    \item[]  $\xrightarrow{2.2}$ \verb|#define C(X) (AR)|
    \item[]  $\xrightarrow{3.1}$ \verb|#define C(X) (F(X)R)|
    \item[]  $\xrightarrow{2.1}$ \verb|#define C(X) (S(X)R)|
    \item[]  $\xrightarrow{4.1}$ \verb|#define C(X) (S(X)*AR)|
    \item[]  $\xrightarrow{3.2}$ \verb|#define C(X) (S(X)*(X)R)|
    \item[]  $\xrightarrow{4.2}$ \verb|#define C(X) (S(X)*(X))|
\end{itemize}

\subsection{Перевірка на непродуктивність}
\begin{enumerate}
    \item  F A R
    \item  F A R I
    \item  нема непродуктивних символів
\end{enumerate}

\subsection{Перевірка на недосяжність}
\begin{enumerate}
    \item  I
    \item  I F A R
    \item  нема недосяжних символів
\end{enumerate}

\subsection{Тип граматики}
Граматика не є простою. Граматика є LL(1) граматикою.


\newpage
\section{Побудова магазинного автомату}
\subsection{Побудова функції ПЕРШ}
\begin{itemize}
    \item  ПЕРШ(\verb|I| $\to$ \verb|#define F(X) (AR)|) = \{\verb|#define|\}
    \item  ПЕРШ(\verb|F| $\to$ \verb|S|) = \{\verb|S|\}
    \item  ПЕРШ(\verb|F| $\to$ \verb|C|) = \{\verb|C|\}
    \item  ПЕРШ(\verb|A| $\to$ \verb|F(X)|) = \{\verb|S, C|\}
    \item  ПЕРШ(\verb|A| $\to$ \verb|(X)|) = \{\verb|(|\}
    \item  ПЕРШ(\verb|R| $\to$ \verb|*AR|) = \{\verb|*|\}
    \item  ПЕРШ(\verb|R| $\to$ \verb|$|) = \{\verb|$|\}
\end{itemize}

\subsection{Побудова функції СЛІД}
\begin{itemize}
    \item  СЛІД(\verb|I|) = \{\verb|$|\}
    \item  СЛІД(\verb|F|) = \{\verb|(|\}
    \item  СЛІД(\verb|A|) = \{\verb|)|\} $\cup$ ПЕРШ(\verb|R|) = \{\verb|)|, \verb|*|\}
    \item  СЛІД(\verb|R|) = \{\verb|)|\}

\end{itemize}

\subsection{Побудова функції ВИБІР}
\begin{itemize}
    \item  ВИБІР(\verb|I| $\to$ \verb|#define F(X) (AR)|) = ПЕРШ(1) = \{\verb|#define|\}
    \item  ВИБІР(\verb|F| $\to$ \verb|S|) = ПЕРШ(2.1) = \{\verb|S|\}
    \item  ВИБІР(\verb|F| $\to$ \verb|C|) = ПЕРШ(2.2) = \{\verb|C|\}
    \item  ВИБІР(\verb|A| $\to$ \verb|F(X)|) = ПЕРШ(3.1) = \{\verb|S, C|\}
    \item  ВИБІР(\verb|A| $\to$ \verb|(X)|) = ПЕРШ(3.2) = \{\verb|(|\}
    \item  ВИБІР(\verb|R| $\to$ \verb|*AR|) = ПЕРШ(4.1) = \{\verb|*|\}
    \item  ВИБІР(\verb|R| $\to$ \verb|$|) = СЛІД(R) = \{\verb|)|\}
\end{itemize}


\newpage
\subsection{Побудова команд розпізнавача}
\begin{enumerate}
    \item  f (\verb|s, #define, I |) = \verb|(s, )RA()X(F )|\
    \item  f (\verb|s, S, F |) = \verb|(s, $)|\
    \item  f (\verb|s, C, F |) = \verb|(s, $)|\
    \item  f* (\verb|s, S, A |) = \verb|(s, )X(F )|\
    \item  f* (\verb|s, C, A |) = \verb|(s, )X(F )|\
    \item  f (\verb|s, (, A |) = \verb|(s, )X )|\
    \item  f (\verb|s, *, R |) = \verb|(s, RA)|\
    \item  f* (\verb|s, ), R |) = \verb|(s, $)|\
    \item  f (\verb|s, (, ( |) = \verb|(s, $)|\
    \item  f (\verb|s, ), ) |) = \verb|(s, $)|\
    \item  f (\verb|s, X, X |) = \verb|(s, $)|\
    \item  f* (\verb|s, $, h0 |) = \verb|(s, $)|\
\end{enumerate}

\newpage
\subsection{Перевірка команд розпізнавача}
Ланцюжок: \verb|#define C(X) (S(X)*(X))|

\begin{itemize}
    \item[]  (s, \quad \verb|#define C(X) (S(X)*(X))$|,           \quad $h_{0}$\verb|I|)      $\vdash$ 1
    \item[]  (s, \quad \verb|C(X) (S(X)*(X))$|,           \quad $h_{0}$\verb|)RA()X(F |)      $\vdash$ 3
    \item[]  (s, \quad \verb|(X) (S(X)*(X))$|,           \quad $h_{0}$\verb|)RA()X( |)        $\vdash$ 9
    \item[]  (s, \quad \verb|X) (S(X)*(X))$|,           \quad $h_{0}$\verb|)RA()X |)          $\vdash$ 11
    \item[]  (s, \quad \verb|) (S(X)*(X))$|,           \quad $h_{0}$\verb|)RA() |)            $\vdash$ 10
    \item[]  (s, \quad \verb|(S(X)*(X))$|,           \quad $h_{0}$\verb|)RA( |)               $\vdash$ 9
    \item[]  (s, \quad \verb|S(X)*(X))$|,           \quad $h_{0}$\verb|)RA |)                 $\vdash$ 4
    \item[]  (s, \quad \verb|S(X)*(X))$|,           \quad $h_{0}$\verb|)R)X(F |)              $\vdash$ 2
    \item[]  (s, \quad \verb|(X)*(X))$|,           \quad $h_{0}$\verb|)R)X( |)                $\vdash$ 9
    \item[]  (s, \quad \verb|X)*(X))$|,           \quad $h_{0}$\verb|)R)X |)                  $\vdash$ 11
    \item[]  (s, \quad \verb|)*(X))$|,           \quad $h_{0}$\verb|)R) |)                    $\vdash$ 10
    \item[]  (s, \quad \verb|*(X))$|,           \quad $h_{0}$\verb|)R |)                      $\vdash$ 7
    \item[]  (s, \quad \verb|(X))$|,           \quad $h_{0}$\verb|)RA |)                      $\vdash$ 6
    \item[]  (s, \quad \verb|X))$|,           \quad $h_{0}$\verb|)R)X |)                      $\vdash$ 11
    \item[]  (s, \quad \verb|))$|,           \quad $h_{0}$\verb|)R) |)                        $\vdash$ 10
    \item[]  (s, \quad \verb|)$|,           \quad $h_{0}$\verb|)R |)                          $\vdash$ 8
    \item[]  (s, \quad \verb|)$|,           \quad $h_{0}$\verb|) |)                           $\vdash$ 10
    \item[]  (s, \quad \verb|$|,           \quad $h_{0}$\verb||)                              $\vdash$ 12
    \item[]  (s, \quad \verb|$|,           \quad \verb|$|) - успішне розпізнавання












    







\end{itemize}

Після перевірки ми дійшли до заключної конфігурації, отже рядок належить до граматики.


